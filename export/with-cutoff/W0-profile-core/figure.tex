\begin{figure}%
	\centering%
	\includegraphics[width=\textwidth]{\ROOTPATH/figure.pdf}
	\caption{Cutoff parameter analysis illustrated with density, velocity and mass profiles for $W_0\in\qbracket{17.7,76.4}$, $\beta_0 = 10^{-6}$ and $\theta_0 = 20$. There are three main regimes: (\textbf{left}) for high central cutoff values, $W_0 \gg \theta_0$ the outer halo is described by an lowered isothermal sphere. The surface radius is approx. proportional to $W_0$. The lower $W_0$ the smaller the surface radius (relative to the core radius). (\textbf{middle}) There is a local surface radius minimum when $W_0$ becomes low enough such that the halo gets affected. For decreasing $W_0$ the surface radius begins to increase again and the halo becomes cuspy. This transition from cored to cuspy halo ends when the plateau and halo radii merge to a saddle point in the rotation curve. (\textbf{right}) After the saddle point the surface radius continues to increase a bit more but with a \textit{halo deficit}. Finally, after a local maximum in the cuspy regime the halo gets disrupted for decreasing $W_0$. Note that radii are given in units of the central density $\rho_0$, velocities are given in units of the core velocity (maximum) and masses are given in units of the core mass. In each plot the limiting solution ($W_0 \to \infty)$ is plotted as a solid black line for comparison.}%
	\label{fig:profile:with-cutoff:W0:core}%
\end{figure}