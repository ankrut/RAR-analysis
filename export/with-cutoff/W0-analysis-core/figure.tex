\begin{figure}%
	\centering%
	\includegraphics[width=\textwidth]{\ROOTPATH/figure.pdf}
	\caption{Cutoff parameter analysis illustrated with particular values of the density, mass and degeneracy variables for $\beta_0 = 10^{-6}$ and $\theta_0 = 20$. Those values are obtained at the core, plateau, halo and surface. There are three main regimes: for high central cutoff values, $W_0 \gg \theta_0$ the outer halo is described by an lowered isothermal sphere. The surface radius follows a power law. There is a local surface radius minimum when $W_0$ becomes low enough such that the halo gets affected. It follows a sharp peak characterizing the transition from cored to cuspy halos. This transition ends with the \textit{halo deficit} regime. The particular mass values follow a similar behavior as for the radius except the sharp peak. In the halo deficit regime the total mass is dominated by the core. The different transition points become clear in the analysis of the degeneracy parameter. Note that radii are given in units of the central density $\rho_0$, velocities are given in units of the core velocity (maximum) and masses are given in units of the core mass.}%
	\label{fig:analysis:with-cutoff:W0:core}%
\end{figure}