\begin{figure}%
	\centering%
	\includegraphics[width=\textwidth]{\ROOTPATH/figure.pdf}
	\caption{Degeneracy parameter analysis illustrated with density, velocity and mass profiles for $\theta_0\in\qbracket{-3.6,40}$, $\beta_0 = 10^{-6}$ and $W_0 = 36$. There are three main regimes: (\textbf{left}) for low central degeneracy values, $\theta_0 \ll -1$  the mass distribution is well described by an isothermal sphere with cutoff. The cutoff is characterized through $W_0$. For increasing $\theta_0$ a degenerate core followed by an plateau is clearly formed, in particular for $\theta_0 > 0$. Simultaneously the surface radius and density of the formed plateau decrease. The maximal cored halo (longest plateau) is obtained when the surface radius reaches a local minimum. (\textbf{middle}) For increasing $\theta_0$ the halo gets affected such that the morphology from the isothermal halo to a cored halo continues to a cuspy halo. This transition ends when the plateau and halo radii merge to a saddle point in the rotation curve. (\textbf{right}) After the saddle point the surface radius continues to increase a bit more but with a \textit{halo deficit}. Finally, after a local surface radius maximum in the cuspy regime the halo gets disrupted for increasing $\theta_0$. Note that radii are given in units of the central density $\rho_0$, velocities are given in units of the core velocity (maximum) and masses are given in units of the core mass. In each plot the limiting solution ($\theta_0 \ll -1)$ is plotted as a solid black line for comparison.}%
	\label{fig:profile:with-cutoff:theta0:core}%
\end{figure}